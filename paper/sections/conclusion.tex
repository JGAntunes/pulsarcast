%!TEX root = ../paper.tex

\section{Conclusion}
\label{section:conclusion}

In this work, we introduced Pulsarcast, a decentralised, topic-based, pub-sub
solution that seeks to bring reliability and eventual delivery guarantees
(commonly associated with centralised solutions) to the P2P realm. We analysed
how Pulsarcast provides a feature rich API on top of a system that leverages a
Kadmelia structured overlay to build immutable and content-addressable data
structures (Merkle DAG) representing both topics and events. These structures
power Pulsarcast's eventual delivery guarantees.

We observed that Pulsarcast surpassed IPS's current implementation (Floodsub)
in every aspect, providing a better QoS with a smaller resource
footprint. The only exception being the order guarantee scenarios, however we
are looking at total different levels of QoS. Resource wise, Floodsub is far
more network-intensive than Pulsarcast (with six times more usage in some
cases) and generally requires more CPU power.  It is also essential to consider
Pulsarcast's high publish rates, given that for each event published we store
it in the DHT. This is the cornerstone of its eventual delivery guarantees,
giving applications the ability to fetch missing events from their event tree.

We concluded that our system provides a good alternative to applications that
seek a better QoS level as well as a feature-rich topology setting, that allows
to restrict publishers and configure topics to one's needs. Despite being
heavily reliant on a structured overlay, Pulsarcast did not underperform under
adverse network conditions, making it suitable for multiple scenarios. 
