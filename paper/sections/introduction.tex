%!TEX root = ../paper.tex

\section{Introduction}\label{section:introduction}


The publish-subscribe (pub-sub) interaction paradigm is an approach that has
received an increasing amount of attention over the course of the century
\cite{Kermarrec2013} \cite{Eugster2003}. This is mainly due to its special
properties, that allow for full decoupling of time, space and synchronisation
of all the communicating parties. In this interaction paradigm, subscribers (or
consumers) sign up for events, or classes of events, from publishers (or
producers) that are subsequently asynchronously delivered. Taking a closer look
at this definition one can see that this comes hand in hand with the way
information is consumed nowadays, with the exponential growth of social
networks like Twitter and the usage of feeds such as RSS.

Due to the properties described above, a lot of applications rely on the
publish-subscribe paradigm and a lot of work has been done by companies like
Twitter~\footnote{\url{https://www.infoq.com/presentations/Twitter-Timeline-Scalability}},
Spotify~\cite{Setty2013} and LinkedIn into making these systems capable of
scaling to a large number of participants, with the creation of tools like
Kafka~\footnote{\url{http://kafka.apache.org/documentation/\#design}}, which
aim at guaranteeing low latency and high event throughput. Other examples are
the multiple message queue systems like Apache Active
MQ~\footnote{\url{http://activemq.apache.org}},
RabbitMQ~\footnote{\url{https://www.rabbitmq.com/}},
Redis~\footnote{\url{https://redis.io/topics/pubsub}}, etc. Most of these
solutions are, of course, centralised and as such suffer from all the common
issues that affect centralised solutions: it is quite hard to maintain and
scale these systems to a large number of clients. Peer to peer (P2P) networks,
on the other hand, have proven numerous times, that this is where they shine,
with examples such as Gnutella, Skype and most recently
ipfs~\footnote{\url{https://ipfs.io/}}. All of these systems are a
living proof of the high scalability P2P can offer, with pub-sub
systems over P2P networks being an active research topic with a lot
of attention.

As we are going to cover in the next sections, lots of different solutions
exist. However, most of them either rely on a centralised or hierarchic network
to have a reliable system, with stronger delivery and persistence guarantees,
or end up sacrificing these same properties in order to have a decentralised
system with the potential to scale to a much larger network. There is also, to
the best of our knowledge, a lack of pub-sub systems with a strong focus on
persistence.

We intend to address this in Pulsarcast by focusing in the following
properties:

\begin{itemize}
  \item
    Eventual delivery guarantees;
  \item
    Data persistence;
  \item
    Ability to scale to a vast number of users;
  \item
    Take advantage of the network infrastructure and network protocols we have in place today;
  \item
    Strong focus on reliability;
\end{itemize}

Besides the specification and architectural model of our system we also provide
a concrete implementation of it. So, in order to validate the solution we
purpose we have created the following:

\begin{itemize}
  \item
    A Javascript implementation module of Pulsarcast with a clearly defined API (Application Programming Interface) through which applications can integrate with;
  \item
    A distributed test runner capable of running large scale test scenarios and simulate abnormal network conditions;
  \item
    An easy to automate test-suite based on a real world application;
\end{itemize}

This document is structured as follows: Section \ref{section:related-work}
presents and analyses our related work. Section \ref{section:pulsarcast}
introduces and describes Pulsarcast, its architecture, data structures and
algorithms. Section \ref{section:implementation} covers the implementation of
our solution, with a more thorough overview of our Javascript module. Next,
Section \ref{section:evaluation} explains our evaluation methodology and
presents those results. Finally, Section \ref{section:conclusion} provides a
set of closing remarks and a set of improvements and future work.
