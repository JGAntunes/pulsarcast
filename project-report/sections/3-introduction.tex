%!TEX root = ../report.tex

%
% Introduction
%

\section{Introduction}

The publish-subscribe (pub-sub) interaction paradigm is an approach that
has received an increasing amount of attention recently \cite{Kermarrec2013} \cite{Eugster2003}.
This is mainly due to its special properties, that allow for full decoupling of all the communicating
parties. First we should define what the publish-subscribe pattern is.
In this interaction paradigm, subscribers (or consumers) sign up for
events, or classes of events, from publishers (or producers) that are
subsequently asynchronously delivered. Taking a closer look at this
definition one can see that this comes hand in hand with the way
information is consumed nowadays, with the exponential growth of social
networks like Twitter and the usage of feeds such as RSS.

The previously discussed decoupling can be broken in three different
parts. The decoupling in time, space and synchronisation. The time
decoupling comes from the fact that publishers and subscribers do not
need to be actively interacting with each other at the same time; this
means that the publisher can publish some events while the subscriber is
disconnected and the subscriber can be notified of an event whose
publisher is disconnected. Space decoupling gives both parties the
benefit of not needing to know each other in order to communicate, given
that consumers and producers are focused on they are specific roles
(consuming/producing) and do not care for who is doing what, or how many
producers are for example. Synchronization decoupling is a consequence
of the asynchronous nature of the pub-sub pattern, as publishers do not
need to be blocked while producing events and subscribers can be
asynchronously notified. The decoupling that this kind of system offers
makes it the ideal candidate for very large networks that need a way to
communicate in a efficient way.

Due to the properties described above, a lot of applications rely on the
publish-subscribe paradigm and a lot of work as been done by companies
like Twitter~\footnote{https://www.infoq.com/presentations/Twitter-Timeline-Scalability}
and LinkedIn into making these systems highly scalable, with the
creation of tools like Kafka~\footnote{http://kafka.apache.org/documentation/\#design}, which aim at
guaranteeing low latency and high event throughput. Other examples are
the multiple message queue systems like Apache Active MQ~\footnote{http://activemq.apache.org/},
RabbitMQ~\footnote{https://www.rabbitmq.com/}, Redis~\footnote{https://redis.io/topics/pubsub}, etc.
These solutions are, of course, centralised and as such suffer from all the common issues that
affect centralised solutions: it is quite hard to maintain and scale
these systems to a large number of clients. Peer-to-Peer networks on the
other hand, have proven numerous times, that this is where they shine,
with examples such as Gnutella, Skype and most recently IPFS~\footnote{https://ipfs.io/}.
All of these systems are a living proof of the high
scalability P2P can offer, with pub-sub systems over P2P networks being
an active research topic with a lot of attention.
