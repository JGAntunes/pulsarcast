
%!TEX root = ../dissertation.tex

\begin{otherlanguage}{portuguese}
\begin{abstract}
\abstractPortuguesePageNumber

O paradigma de produtor-consumidor é uma das mais populares formas de
comunicação em sistemas distribuídos complexos. Existe bastante investigação
sobre o mesmo, com soluções desde \emph{brokers} de mensagens centralizados,
até sistemas completamente descentralizados (\emph{peer to peer}). Soluções
descentralizadas apresentam, por norma, a melhor opção quando o foco é
escalabilidade. Contudo, existe uma clara lacuna em sistemas descentralizados
que tenham em conta confiabilidade, garantis de entrega de mensagens e, acima
de tudo, persistência. É com esse mesmo fim que apresentamos o Pulsarcast, um
sistema de comunicação produtor-consumidor, descentralizado, escalável, que
procura dar o tipo de garantias normalmente associadas a arquitecturas
centralizadas tais como persistência e garantia de entrega eventual de
mensagens.

% Keywords
\begin{flushleft}

  \palavrasChave{Produtor Consumidor, \emph{Peer to peer}, Descentralizado, \emph{Web}, Confiabilidade, Persistência, Garantia de entrega}

\end{flushleft}

\end{abstract}
\end{otherlanguage}
